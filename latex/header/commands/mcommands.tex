% Mathematical Functions
% Definitions that are more complex than an alias
% ###################################################

\newcommand*{\nofrac}[2]{#1/#2}

% Linear algebra
% ---------------------------------------

% Small cvec

% transpose
\newcommand*{\trans}{^\mathrm{T}}

% Vectors
\newcommand*{\vect}[1]{\boldsymbol{#1}}

% Wrap environments into commands
\newcommand*{\scvec}[1]{\begin{psmallmatrix}#1\end{psmallmatrix}}
\newcommand*{\cvec}[1]{\begin{pmatrix}#1\end{pmatrix}}
\newcommand*{\pmat}[1]{\begin{pmatrix}#1\end{pmatrix}}
\newcommand*{\mat}[1]{\begin{matrix}#1\end{matrix}}
\newcommand*{\spmat}[1]{\begin{psmallmatrix}#1\end{psmallmatrix}}

% MISC
% ---------------------------------------

% Nonbreaking dash for stuff like $n$-dimensional:
\newcommand*{\ndash}{\nobreakdash-\hspace{0pt}}

% Spacing in Math mode by \back{10}
\newcommand*{\back}[1]{\mspace{-#1mu}}

%% Restriction of a function
%\newcommand*\restr[2]{{% we make the whole thing an ordinary symbol
%  \left.\kern-\nulldelimiterspace % automatically resize the bar with \right
%  #1 % the function
%  \vphantom{\big|} % pretend it's a little taller at normal size
%  \right|_{#2} % this is the delimiter
%  }}


\newcommand{\xdownarrow}[1]{%
	{\left\downarrow\vbox to #1{}\right.\kern-\nulldelimiterspace}
}

% 
% ANALYSIS
% ---------------------------------------

%\newcommand*\ddiff[2]{\frac{\mathrm{d}#1}{\mathrm{d}#2}}
%\newcommand*\pardiff[2]{\frac{\partial#1}{\partial#2}}
%\newcommand*\vardiff[2]{\frac{\delta#1}{\delta#2}}


% SLIGHTLY MORE COMPLICATED STUFF
% --------------------------------

% Kerning of \defeq 
% credit: http://tex.stackexchange.com/questions/4216/how-to-typeset-correctly
% normal
% \newcommand*{\defeq }{:=}
% zentriert (funktioniert nicht bei kleinerer Schriftgröße!!)
% \makeatletter
% \newcommand*{\defeq}{\mathrel{\rlap{%
%                      \raisebox{0.3ex}{$\m@th\cdot$}}%
%                      \raisebox{-0.3ex}{$\m@th\cdot$}}%
%                      =}
% \makeatother
% Für Kleine Punkte (selbe Linienstärke wie "="), einfach
% die beiden "." in den hboxes durch "\scriptsize." ersetzen
\newcommand*{\defeq}{
                      \mathrel{\vcenter{
                         \baselineskip0.5ex 
                         \lineskiplimit0pt
                         \hbox{.}
                         \hbox{.}
                      }}
                      =
                    }
\newcommand*{\eqdef}{ =
                      \mathrel{\vcenter{
                         \baselineskip0.5ex 
                         \lineskiplimit0pt
                         \hbox{.}
                         \hbox{.}
                      }} 
                    }
