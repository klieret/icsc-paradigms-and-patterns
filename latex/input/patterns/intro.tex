\begin{frame}
	\tableofcontents[]
	Slides + exercises available at \hhl{\href{https://github.com/klieret/icsc-paradigms-and-patterns}{github.com/klieret/icsc-paradigms-and-patterns}}
\end{frame}

\begin{frame}[t]{Repetition: Object Oriented Programming}
	\begin{itemize}[<+->]
		\item \hhl{Inheritance:} Subclasses inherit all \srem{(public and protected)} attributes and methods of the base class
		\item Methods and attributes can be \hhl{public} \srem{(anyone has access)}, \hhl{private} \srem{(only the class itself has access)} or \hhl{protected} \srem{(only the class and its subclasses have access)}
		\item \hhl{Abstract methods} of an \hhl{abstract class} are methods that have to be implemented by a subclass (\hhl{concrete class})
	\end{itemize}

	\only<+->{
		\bigskip
		\stress{New}: \hhl{class methods}: Methods of the class, rather than its instances
		\inputminted[fontsize=\small]{python}{code/patterns/intro/classmethod.py}
	}
\end{frame}


\section{More on OOP}

\subsection{Class Diagrams}

\begin{frame}{Class diagrams I}
	UML \srem{(\underline Unified \underline Markup \underline Language)} class diagrams visualize classes and the relationships between them. We will use the following subset of notations:

	\bigskip
	\begin{changemargin}{-1cm}{-1cm}
	\centering
	\begin{tikzpicture}[scale=0.9, transform shape]
	\umlclass[x=0,y=3]{class A}{
		+ public variable : type\\
		-- private variable : type \\
		\# protected variable : type
	}{
		+ public method(arg : type) : return type\\
		\dots
	}

	\umlemptyclass[x=0,y=0]{class B}
	\umlnote[x=3,y=0, width=2cm, scale=0.9, transform shape]{class B}{\ subclass of A}
	\umlinherit{class B}{class A}


	\umlemptyclass[x=0,y=-2]{class C}
	\umlnote[x=3,y=-2,width=2cm, scale=0.9, transform shape]{class C}{\ subclass of B}
	\umlinherit{class C}{class B}

	\umlemptyclass[x=6, y=3]{class D}
	\umlaggreg{class A}{class D}
	\umlnote[x=9,y=3,width=2cm, scale=0.9, transform shape]{class D}{\ A contains\\\ instances of D}

	\umlemptyclass[x=6, y=1]{class E}
%	\umlaggreg{class A}{class D}
	\umlnote[x=9,y=1,width=2cm, scale=0.9, transform shape]{class E}{\ A is otherwise associated with E}
	\umlassoc{class E}{class A}
%

	\end{tikzpicture}
	\end{changemargin}
\end{frame}


\subsection{The SOLID rules of OOP}

%\begin{frame}{The SOLID rules of OOP: SOLID}
%
%\end{frame}

\frame{\tableofcontents[currentsection, currentsubsection]}

\begin{frame}[t]{The SOLID rules of OOP: Single responsibility principle}{\underline SOLID}
	\bigskip
	% https://hackernoon.com/you-dont-understand-the-single-responsibility-principle-abfdd005b137
	Commonly (mis-)quoted as:

	\begin{quote}
		A class should only have \hhl{one responsibility}.
	\end{quote}

	{
		\medskip
		More accurate:

		\begin{quote}
			A class should only have \hhl{one reason to change}.
		\end{quote}
	}

	\only<2->{
		\medskip
		Better still:
		\begin{quote}
			\hhl{Gather} together the things that change for the same reasons. \\
			\hhl{Separate} those things that change for different reasons.
		\end{quote}

		\only<3->{
			\medskip
			So this actually proposes a \hhl{balance}!
			\begin{itemize}
				\item Avoid classes that do too much \srem{(\enquote{god class})}
				\item But also avoid having changes always affect many classes \srem{(\enquote{shotgun surgery})}
			\end{itemize}
		}
	}
\end{frame}

\begin{frame}{The SOLID rules of OOP: Open Closed Principle}{S\underline  OLID}
	\begin{quote}
		You should be able to \hhl{extend} the behavior of a system\\
		without having to \hhl{modify} that system.
	\end{quote}

	\begin{itemize}
		\item<2-> Writing a library, \hhl{modifying} functionality means that all users have to be informed (\hhl{not backwards compatible}) $\lra$ Avoid!
		\item<2-> In your own code: Modifying one functionality \srem{(also by overriding methods of the super class, etc.)} poses the danger of breaking other parts \srem{(though tests can help with that)}
		\item<3-> Extending code by providing additional methods, attributes, etc. does not have this danger $\lra$ preferred!
		\item<4-> Requires thinking ahead: What parts have to be flexible, what remains constant?
		\item<4-> Again a \hhl{balance} is required:
		\begin{itemize}
			\item Be \hhl{too generic} (avoid modifications) and your code won't do anything
			\item Be \hhl{too concrete} and you will need to modify \srem{(and potentially break things)} often
		\end{itemize}
	\end{itemize}
\end{frame}

\begin{frame}[t]{The SOLID rules of OOP: Liskov Substitution Principle}{SO\underline  LID}
	\begin{quote}
		If $S$ is a \hhl{subtype} (subclass) of $T$, then objects of type $T$ can be \hhl{replaced} with objects of type $S$ without breaking anything
	\end{quote}

	\srem{e.g. I can replace all instances of \texttt{Animal} with instances of \texttt{Cat}}

	\medskip
	\only<2->{
		This can be expanded to a series of properties that should be fulfilled:
		\begin{itemize}[<+->]
			\item \hhl{Signature} of methods of the subclass:
			\begin{itemize}[<+->]
				%
				\item
				Required type of arguments should be supertype (\defn{contravariance})\\
				\srem{Violation: Supermethod accepts any \texttt{Animal}, submethod only \texttt{Cat}}
				%
				\item
				Return type of method should be a subtype (\defn{covariance})\\
				\srem{Violation: Supermethod returns \texttt{Cat}, submethod returns \texttt{Animal}}
			\end{itemize}
		\item Behavior:
			\begin{itemize}[<+->]
			%
			\item
			\hhl{Preconditions} \srem{(requirements to be fulfilled before calling method)} cannot be strengthened in the subtype\\
			\srem{Violation: Only in subclass \texttt{prepare()} must be called before \texttt{method()}}
			%
			\item
			\hhl{Postconditions}
			\srem{(conditions fulfilled after calling a method)} cannot be weakened by the subtype
			%
			\item
			\hhl{Invariants} \srem{(properties that stay the same)} of supertype must be preserved in the subtype
			%
			\item
			\hhl{History constraint}: Subtypes cannot modify properties that are not modifiable in supertype\\
			\srem{Violation: \texttt{VariableRadiusCircle} as subtype of \texttt{FixedRadiusCircle}}
			\end{itemize}
		\end{itemize}
	}
\end{frame}

\begin{frame}{The SOLID rules of OOP: Interface segregation principle (ISP)}{SOL\underline  ID}
	\begin{quote}
		Clients should not be forced to depend on methods they do not use
	\end{quote}

	\bigskip
	\begin{itemize}[]
		\item<2->
		\hhl{\enquote{Thin} interfaces} offering a reasonably small number of methods with \defn{high cohesion} \srem{(serve similar purposes; belong logically together)} are \stress{preferred} over \hhl{\enquote{fat} interfaces} offering a large number of methods with low cohesion
		\item<3->
		Sometimes we should therefore split up (\hhl{segregate}) fat interfaces into thinner \hldefn{role interfaces}
		\item<3->
		This leads to a more \hhl{decoupled} system that is easier to maintain
		\item<4-> \hhl{Example}: Even if all data is contained in one \srem{(e.g. SQL)} database, the ISP asks to write \emph{different} interfaces to do different things, e.g. have a \texttt{CustomerDb}, \texttt{OrderDb}, \texttt{StoreDb}, \dots
	\end{itemize}
\end{frame}

\begin{frame}[t]{The SOLID rules of OOP: Dependency Inversion Principle}{SOLI\underline  D}
	This is about \hhl{decoupling} different classes and modules:
	\begin{quote}
		\begin{itemize}
			\item[1.] High-level modules should not depend on low-level modules. Both should depend on abstractions \srem{(interfaces)}.
		\end{itemize}
	\end{quote}
	\only<2->{
		Let's consider a very simple example: A \hhl{button} controlling a lamp. One way to implement this:

		\bigskip
		\begin{columns}
			\column{0.1\textwidth}
			%
			\begin{tikzpicture}[scale=0.7, transform shape]
			\umlclass[x=0,y=3]{Button}{
				- client
			}{
				+ turn\_on()\\
				+ turn\_off()\\
				+ flip()
			}

			\umlemptyclass[x=0,y=0]{Lamp}
			\umlaggreg{Button}{Lamp}
			\end{tikzpicture}

			\column{0.8\textwidth}
			\only<3->{
				This violates the DIP, because \texttt{Button} \srem{(high-level)} depends on \texttt{Lamp} \srem{(detail)}.
				\bigskip

				What if we have multiple consumers \srem{(\texttt{Motor}, \texttt{Lamp}, \dots)} and multiple types of buttons \srem{(swipe button, switch, push button, \dots)}? How can we force them to behave consistent? What methods does a consumer have to implement to work together with the button?\\

				\bigskip
				$\lra$ Enter abstractions (interfaces)
			}
		\end{columns}
	}
	%
\end{frame}

\begin{frame}[t]{The SOLID rules of OOP: Dependency Inversion Principle}{SOLI\underline  D}
	\bigskip
	\begin{center}
	\begin{tikzpicture}[scale=0.7, transform shape]
	\umlclass[x=0,y=3, type=abstract]{AbstractButton}{
		- client
	}{
		+ turn\_on()\\
		+ turn\_off()\\
		+ flip()
	}

	\umlemptyclass[x=0,y=0]{ConcreteButton}
	\umlinherit{ConcreteButton}{AbstractButton}

	\umlclass[y=3,x=5, type=abstract]{AbstractClient}{
	}{
		+ turn\_on()\\
		+ turn\_off()\\
	}
	\umlaggreg{AbstractButton}{AbstractClient}

	\umlemptyclass[y=0,x=5]{ConcreteClient}
	\umlinherit{ConcreteClient}{AbstractClient}
	\end{tikzpicture}
	\end{center}

	\only<2->{
		\medskip
		Now it's clear which methods the concrete client has to implement. Both high level and low level modules only depend on abstractions.

		\bigskip
		\only<3->{
			This also fulfills the second part of the DIP:
			\begin{quote}
				\begin{itemize}
					\item[2.] Abstractions should not depend on details. Details \srem{(i.e. concrete implementations)} should depend on abstractions.
				\end{itemize}
			\end{quote}
		}
	}
\end{frame}

\begin{frame}[t]{Performance considerations}

	\only<1->{
		Some patterns will advocate:
		\begin{itemize}
			\item Classes that only act as interfaces and pass on calls to other (worker) classes
			\item Using separate classes to facilitate communication between classes
			\item Accessing attributes (only) through methods
			\item Prefer composition over inheritance
		\end{itemize}
	}

	\only<2->{
		\bigskip
		However, when writing \hhl{performance critical} (\proglang{C++}, \dots) code, you should avoid unnecessary \enquote{detours}:

		\begin{itemize}
			\item Avoid unnecessary interfaces
			\item Consider inlining simple, often-called functions \srem{(e.g. getters and setters)}
			\item Inheritance > composition > \texttt{if} statements
		\end{itemize}
		Modern compilers will try to apply some optimization techniques automatically \srem{(automatic inling, return value optimization, \dots)}

		\bigskip
		General rule: \hl{Profile before Optimizing}
	}
	%
\end{frame}

\section{Patterns}

\frame{\tableofcontents[currentsection, currentsubsection]}

\subsection{Patterns}

\begin{frame}[t]{Patterns}
	\only<+->{
			\bigskip
		\begin{quote}
			Software design patters try to offer \hhl{general} and \hhl{reusable} solutions for \hhl{commonly} occurring problems in a given \hhl{context}.
		\end{quote}

		\bigskip
	}

	\only<+->{
		Commonly categorized as:

		\medskip
		\begin{itemize}[<+->]
			%
			\item
			\hhl{Creational patterns}: How are instances of classes instantiated? \srem{(What if I have a class that can create instances in different ways?)}
			%
			\item
			\hhl{Structural patterns}: Concerned with relationships between classes. \srem{(How can classes form flexible larger structures?)}
			%
			\item
			\hhl{Behavioral patterns}: Concerned with algorithms and communication between classes. \srem{(How are responsibilities assigned between classes?)}
			%
			\item
			\hhl{Parallel patterns}: Parallel processing and OOP \srem{$\lra$ only mentioned briefly}
		\end{itemize}
	}
\end{frame}
%
%
%
