\subsection{Concurrency Patterns}

\frame{\tableofcontents[currentsection, currentsubsection]}

\begin{frame}{Concurrency Patterns}
	\begin{quote}
		Concurrency patterns need their own lecture, so this will only quickly mention basic concepts.
	\end{quote}

	\bigskip
	Use cases:
	\begin{itemize}[]
		\item<2-> Manage/synchronize access to \hhl{shared resources} \srem{(e.g. to avoid race conditions when several threads perform read and write operations)}
		\item<3-> \hhl{Scheduling} tasks in parallel
%		\begin{itemize}
%			\item {Thread pool}
%			\item {Compute kernel}
%			\item \dots
%		\end{itemize}
		\item<4-> (A)synchronous \hhl{event handling}
	\end{itemize}
\end{frame}

\begin{frame}{Concurrency Patterns II}
	Advanced example: \hhl{Active object} pattern

	\begin{itemize}[<+->]
		\item Want to decouple method calling from method execution
		\item Can request a calculation early and check later whether the result is available
	\end{itemize}

	\bigskip
	The pattern consists of multiple components:
	\begin{itemize}[<+->]
		\item The \hhl{client} calls a method of a \hhl{proxy}, which (immediately) returns a \hhl{future} object \srem{(can be used to check if results are available and get them)}
		\item At the same time the proxy turns the method call into a \hhl{request} object and adds it to a \hhl{request queue}
		\item A \hhl{scheduler} takes requests from the request queue and executes it \srem{(on some thread)}
		\item Once the request is executed, the result is added to the future object
		\item Only when the client accesses the future \srem{(wants to get the result value)}, the client thread waits \srem{(if the result is already available by that time, no waiting/blocking occurs)}
	\end{itemize}
\end{frame}
